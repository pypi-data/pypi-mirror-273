
The examples assume the following environment variable settings:
\begin{itemize}
\item {\tt PATH} includes {\tt \$DEST/bin}
\item {\tt PYTHONPATH} includes {\tt \$DEST/python}
\end{itemize}
The python examples assume that one has done:
\begin{python}
>>> import seal
\end{python}
A first test:
\begin{python}
>>> seal.hello()
Hello.  This is Seal ...
\end{python}
The actual output will have the version number in place of {\tt XX.YY.ZZ}.

\subsection{Installation}

The ``pure Python'' subset of Seal requires no installation.  One only
needs to assure that {\tt \$SEAL/python} is included on one's
{\tt PYTHONPATH}.

For full functionality, go to the {\tt \$SEAL} directory and do:
\begin{myverb}
$ ./configure
$ make
\end{myverb}
Installation in an external directory is not necessary, provided that
{\tt \$SEAL/python} is included on {\tt PYTHONPATH}.  If installation
in an external directory is desired, one may specify it when calling
{\tt configure}:
\begin{myverb}
$ ./configure --prefix=/usr/local
$ make
$ make install
\end{myverb}
%$
