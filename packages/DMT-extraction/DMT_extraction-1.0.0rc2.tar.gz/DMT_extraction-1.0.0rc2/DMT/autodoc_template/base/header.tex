% DMT
% Copyright (C) from 2022  SemiMod
% Copyright (C) until 2021  Markus Müller, Mario Krattenmacher and Pascal Kuthe
% <https://gitlab.com/dmt-development/dmt-extraction>
%
% This file is part of DMT-extraction.
%
% DMT-extraction is free software: you can redistribute it and/or modify it under the terms of
% the GNU General Public License as published by the Free Software Foundation,
% either version 3 of the License, or (at your option) any later version.

% DMT-extraction is distributed in the hope that it will be useful, but WITHOUT ANY
% WARRANTY; without even the implied warranty of MERCHANTABILITY or
% FITNESS FOR A PARTICULAR PURPOSE. See the GNU General Public License for
% more details.

% You should have received a copy of the GNU General Public License along with DMT-extraction.
% If not, see <https://www.gnu.org/licenses/>.

% This Tex file sets the documentclass and packages for the DMT autodoc template.

\documentclass[english, cd=nocolor, cdfont=off]{tudscrartcl}
\renewcommand{\autodot}{}
%\usepackage{selinput} \SelectInputMappings{adieresis={ä},germandbls={ß}}

% font types:
% times new roman: with serifs
\usepackage{times}
%\usepackage{mathptmx}
\fontfamily{ptm}
\fontsize{12pt}{15pt}
\selectfont
% official TUD font: opensans:
%\usepackage[default]{opensans}

% Line breaks and more
\usepackage[activate]{microtype}

% Distance between text lines
\usepackage{setspace}
\onehalfspacing
%\setstretch{1.25}

\usepackage[toc,page]{appendix}

% Localization
\usepackage[english]{babel}

% Slashboxes in tables
\usepackage{diagbox}

% German quotes ""
\usepackage[babel]{csquotes}

% Graphics
\usepackage{graphicx}
\graphicspath{\base/fig/}
\usepackage{subfiles}
% Mathematical text symbols
\usepackage{amsmath}
\usepackage{amssymb}
\numberwithin{equation}{section}
\renewcommand{\theequation}{\thesection-\arabic{equation}}
\usepackage[integrals]{wasysym} % fuer gerades Integralzeichen
\usepackage{mathtools}
\usepackage{enumitem}
% Fractions in text
\usepackage{nicefrac}
\DeclareMathOperator{\sign}{sign}
% Colors
\usepackage[dvipsnames]{xcolor}

% Links to Figures and not to captions
\usepackage[hypcap]{caption}

% for placing figures side by side
\usepackage{subcaption}
%\usepackage{subfigure}

% for SI units locale=EN,
\usepackage{siunitx}
\sisetup{range-units=repeat, list-units=repeat, binary-units, exponent-product = \cdot}
\usepackage{textcomp} % for having the right \micro symbol in siunitx
\DeclareSIUnit[number-unit-product = {}] \spezOhm{\SIUnitSymbolOhm\metre} % ohm meter
\DeclareSIUnit[number-unit-product = {}] \degC{\degreeCelsius} % degree celsius
\DeclareSIUnit[number-unit-product = {}] \degC{\degree_Celsius} % degree celsius

\DeclareSIUnit\sq{\ensuremath{\Box}}

% for nice tables:
\usepackage{booktabs}
\usepackage{collcell}
\usepackage{multirow}
\usepackage{makecell}
\usepackage{longtable}
\usepackage{array}
\newcolumntype{L}[1]{>{\raggedright\let\newline\\\arraybackslash\hspace{0pt}}m{#1}}

% For drawing circuits
\usepackage[european, siunitx]{circuitikz}
\usetikzlibrary{circuits.ee.IEC} % zeichne elektrische Schaltungen nach ee IEC

\usepackage{tikz}
\usetikzlibrary{external} % lagere Tikz-Bilder nach dem erstellen aus und verwende sie wieder
\tikzexternalize[prefix=figures/, mode=list and make] % activate with path figures/
% \tikzexternalize[prefix=figures/] % activate with path figures/
%using multi core to build pictures
%https://tex.stackexchange.com/questions/136658/how-to-make-pdflatex-multithreaded-when-shell-escape-is-enabled

\usepackage{pgfplots}
\pgfplotsset{compat=newest}
\usetikzlibrary{shapes.geometric}
%% the following commands are needed for some matlab2tikz features
\usetikzlibrary{plotmarks}
\usetikzlibrary{arrows.meta}
\usepgfplotslibrary{patchplots}
\usepackage{grffile}
\pgfplotsset{plot coordinates/math parser=false}
\newlength\figureheight
\newlength\figurewidth
\usepackage{tikzscale}
\usepackage{scrhack}
\usepackage[export]{adjustbox}
\usepackage[normalem]{ulem}
% to insert source code ( hier inp.file)

\DeclareFixedFont{\ttb}{T1}{txtt}{bx}{n}{8} % for bold
\DeclareFixedFont{\ttm}{T1}{txtt}{m}{n}{8}  % for normal
\usepackage{listings}

% Floatbarrier, floats do not randomly goe over sections
\usepackage[section]{placeins}

% for bibtex
\usepackage[sorting=none , backend=bibtex, style=ieee, doi=false, url=false]{biblatex}
\addbibresource{bib.bib}

%path to images
\graphicspath{{images/}{fig/}}

\usepackage{stackengine}
% the PDF settings
%\PassOptionsToPackage{hyphens}{url}
\usepackage[
	pdfauthor={_author},
	pdfpagemode=UseOutlines, % while opening show contents
	pdflang=eng, % language of documents
]{hyperref}
\usepackage{cleveref}
\crefname{equation}{Equ.}{Equs.}
\crefname{figure}{Fig.}{Figs.}
\let\oldref\ref
\renewcommand{\ref}[1]{(\oldref{#1})}

% just one latex run to get new contents
\usepackage{bookmark}

% Abbreviations and Glossaries
\usepackage[nomain, section=section, toc]{glossaries} %
\usepackage{glossary-longbooktabs}

\newglossary[pml]{parameterlist}{slr}{pmn}{List of extracted paramters}
\setglossarystyle{long3col-booktabs}

\makeglossaries

% Hurenkinder und Schusterjungen verhindern (German)
% http://projekte.dante.de/DanteFAQ/Silbentrennung
\clubpenalty=10000
\widowpenalty=10000
\displaywidowpenalty=10000

% (Color-)settings for links in PDF
\hypersetup{%
	colorlinks= true, 			 
	linkcolor = black, %cddarkblue!100,
	citecolor = black, %cddarkblue!100,
	filecolor = black, %cddarkblue!100,
	menucolor = black, %cddarkblue!100,
	urlcolor  = black, %cddarkblue!100,
	bookmarksnumbered=true 
}
